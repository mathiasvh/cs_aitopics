%%%%%%%%%%%%%%%%%%%%%%%%%%%%%%%%%%%%%%%%%%%%%%%%%%%%%%%%%%%%%%%%%%%%%%%%%%%%%%%%
%2345678901234567890123456789012345678901234567890123456789012345678901234567890
%        1         2         3         4         5         6         7         8

% Comment this line out if you need a4paper
\documentclass[letterpaper, 10 pt, conference]{ieeeconf}
% Use this line for a4 paper
%\documentclass[a4paper, 10pt, conference]{ieeeconf}      
% This command is only needed if you want to use the \thanks command
\IEEEoverridecommandlockouts                              
\overrideIEEEmargins
% See the \addtolength command later in the file to balance the column lengths
% on the last page of the document



% The following packages can be found on http:\\www.ctan.org
%\usepackage{graphics} % for pdf, bitmapped graphics files
%\usepackage{epsfig} % for postscript graphics files
%\usepackage{mathptmx} % assumes new font selection scheme installed
%\usepackage{times} % assumes new font selection scheme installed
%\usepackage{amsmath} % assumes amsmath package installed
%\usepackage{amssymb}  % assumes amsmath package installed

\title{\LARGE \textbf{
Capita selecta: AI Topics} \\ Automatic Statistician 
}


\author{Rick van Hek, Mathias Van Herreweghe
}


\usepackage[T1]{fontenc}
\usepackage{graphicx}
\usepackage{lipsum}
\usepackage{verbatim}
\usepackage{amsbsy}
\usepackage{amsmath}
\usepackage[colorinlistoftodos]{todonotes}
\let\proof\relax
\let\endproof\relax
\usepackage{amsthm}
\newtheorem{theorem}{Theorem}

\begin{document}

\maketitle
\thispagestyle{empty}
\pagestyle{empty}


%%%%%%%%%%%%%%%%%%%%%%%%%%%%%%%%%%%%%%%%%%%%%%%%%%%%%%%%%%%%%%%%%%%%%%%%%%%%%%%%
\begin{abstract}

\begin{comment} % serves as an example abstract
In this paper I demonstrate a novel design for an optoelectronic State Machine which replaces input/output forming logic found in conventional state machines with BDD based optical logic while still using solid state memory in the form of flip-flops in order to store states. This type of logic makes use of waveguides and ring resonators to create binary switches. These switches in turn can be used to create combinational logic which can be used as input/output forming logic\cite{Killick2012} for a state machine. Replacing conventional combinational logic with BDD based optical logic allows for a faster range of state machines that can certainly outperform conventional state machines as propagation delays within the logic described are in the order of picoseconds as opposed to nanoseconds in digital logic.
\end{comment}

\end{abstract}

\tableofcontents
%%%%%%%%%%%%%%%%%%%%%%%%%%%%%%%%%%%%%%%%%%%%%%%%%%%%%%%%%%%%%%%%%%%%%%%%%%%%%%%%
\section{Introduction}
\lipsum[1-2]



\section{Results}
\section{Conclusion}
\lipsum[100-101]


\addtolength{\textheight}{-12cm}  

\bibliography{references}
\bibliographystyle{IEEEtran}

\end{document}
